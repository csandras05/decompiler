\chapter{Bevezetés}
\label{ch:intro}

A fordítóprogramok feladata a magas szintű programkód átalakítása a számítógép számára értelmezhető formára. Céljuk, hogy a programozó sokkal magasabb absztrakciós szinten fejezhesse ki szándékát, ezáltal megkönnyítve a szoftverfejlesztés folyamatát. A kódvisszafejtő programok működése ezzel ellentétes, az alacsony szintű, gépközeli kódot alakítják át magas szintű kóddá. Segítségével beleláthatunk a program forráskódjába akkor is, ha csak egy futtatható állomány áll rendelkezésre, ezáltal könnyebben megvédhetjük gépünket a kártékony szoftverektől.

A fenti feladat megoldására már léteznek különböző kódvisszafejtő programok \cite{ghidra},\cite{binaryninja}, ugyanakkor ezek legnagyobb hátránya, hogy a fordítóprogramokhoz hasonlóan bonyolult szabályok alapján működnek. Ez azért probléma, mert ezen szabályokat minden programozási nyelv esetén külön meg kell fogalmazni, ami egy bonyolult és időigényes feladat.

Dolgozatomban egy olyan programot mutatok be, ami a fenti problémát a természetes nyelvfeldolgozásban használt gépi tanulási eszközökkel oldja meg. A neurális gépi fordítás az utóbbi években hatalmas fejlődésen ment keresztül[bert], így könnyen adódik, hogy ezen eredményeket fel lehetne használni a kódvisszafejtéshez, annyi különbséggel, hogy nem angolról németre, hanem például assembly-ről C-re fordítunk.\footnote{Majd látni fogjuk, hogy a programozási nyelvek sajátos szerkezete miatt sajnos nem alkalmazhatók egy az egyben a természetes nyelvek fordítása során elért eredmények.} Ezen módszer előnye, hogy nem szükséges hozzá programozók hosszú ideig munkája a különböző szabályrendszerek megalkotásához, valamint egy másik nyelvre való áttérés sem okoz különösebb nehézséget. Továbbá a különböző programkód generáló szoftvereknek\cite{??} hála, lényegében korlátlan mennyiségű adat áll rendelkezésre.
