\chapter{Fejlesztői dokumentáció}
\label{ch:impl}

\section{Jax & Flax}
Ma már rengeteg különböző magas szintű könyvtár érhető el a \texttt{Python} nyelvhez, melyek megkönnyítik a mélytanulási algoritmusok implementálást. Én ezek közül a Google által fejlesztett \texttt{Jax}-et\cite{jax2018github} és az arra épülő \texttt{Flax}-et\cite{flax2020github} használtam.

\subsection{Jax}
A Jax könyvtár két legfőbb komponense a gyorsított lineáris algebra (\texttt{XLA}) és az automatikus differenciálás. Mindkettő nagyban hozzájárul ahhoz, hogy könnyebben és gyorsabban lehessen különböző gépi tanulási algoritmusokat és modelleket implementálni, ezzel gyorsítva az ezirányú kutatásokat.

A gépi tanulási algoritmusok során nagyon sok lineáris algebrai műveletet (pl. mátrixszorzás) kell végezni, így elenghetetlen, hogy ezeket minél gyorsabban elvégezzük. Ebben segít az \texttt{XLA}, ami egy doménspecifikus fordító, célja, segítségével a lineáris algebrai műveletek elvé 

\subsection{Flax}

\cite{palmtree}