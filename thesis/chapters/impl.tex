\chapter{Fejlesztői dokumentáció}
\label{ch:impl}

A főprogram (betanított modell használata) alapvetően nézet-modell
architektúrát követ.

\section{Felhasználói felület}
A grafikus és a konzolos felhasználói felületek a \texttt{ui} könyvtárban
érhetőek el. A konzolos interfész egy nagyon egyszerű kommunikációt biztosít:
bekéri a felhasználótól a fájl nevét, majd az eredményt struktúrálva kiírja
a konzolra.

A grafikus felület ezzel szemben komplexebb. Az implementálás során
a \texttt{PySimpleGUI}\cite{TODO} könyvtárat használtam, melynek segítségével
könnyen lehet grafikus felületű alkalmazásokat készíteni Python nyelven.
A \texttt{GUI} osztály konstruktorában történik meg a felhasználói felület
elemeinek létrehozása, melyek a következők:
\begin{itemize}
    \item 2 Szövegdoboz (\texttt{Multiline}): a baloldali jeleníti meg az
    eredeti Assembly kódot, a jobboldali a visszafejtés 3 lépését:
    a szegmentált Assembly-t, a maszkolt C és a rekonstruált C-t.
    \item Szövegbeviteli mező (\texttt{TextInput}): Itt lehet megadni
    a betölteni kívánt fájl nevét. Ez alapvetően inaktív, a szövegbevitelt egy
    fájlkereső biztosítja.
    \item Fájlkereső (\texttt{FileBrowswer}): Ez
    \item Gomb: 
\end{itemize}

\section{Gépi tanulási modellek}

\subsection{Jax \& Flax}
Ma már rengeteg különböző magas szintű könyvtár érhető el a \texttt{Python} nyelvhez, melyek megkönnyítik a mélytanulási algoritmusok implementálást. Én ezek közül a Google által fejlesztett \texttt{Jax}-et\cite{jax2018github} és az arra épülő \texttt{Flax}-et\cite{flax2020github} használtam.

A Jax könyvtár két legfőbb komponense a gyorsított lineáris algebra (\texttt{XLA}) és az automatikus differenciálás. Mindkettő nagyban hozzájárul ahhoz, hogy könnyebben és gyorsabban lehessen különböző gépi tanulási algoritmusokat és modelleket implementálni, ezzel gyorsítva az ezirányú kutatásokat.

A gépi tanulási algoritmusok során nagyon sok lineáris algebrai műveletet (pl. mátrixszorzás) kell végezni, így elenghetetlen, hogy ezeket minél gyorsabban elvégezzük. Ebben segít az \texttt{XLA}, ami egy doménspecifikus fordító, célja, segítségével a lineáris algebrai műveletek elvé 

\subsection{tanítási folyamat}
A tanítási folyamat leírását a \texttt{model/train/training.py} fájl
taralmazza. A szegmentálás és a fordítás tanítása során is ezen modul
metódusait hívja a program.

A \texttt{train_step} metódus egy tanítási lépés folyamatát írja le.

\cite{palmtree}
