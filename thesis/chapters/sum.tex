\chapter{Összegzés}
\label{ch:sum}
Dolgozatomban egy olyan programot mutattam be, melynek segítségével lehetőség van
a bináris állományból az eredeti C kód visszaállítására neurális hálók segítségével.
A kódvisszafejtés három fő lépésben történik, ezek közül az első kettő használ
neurális hálókat.

Előkészítésként a futtaható állományból egy bináris elemzővel kinyerjük az Assembly kódot.
Ezután az első lépés, hogy ezt blokkokra bontjuk, ahol egy blokk egy sor C kódnak felel meg,
itt a blokkok meghatározását végzi egy neurális háló. Második lépésként ezen blokkokból előállítunk
egy sablon kódot, ahol a változók és a számliterálok konkrét értékei még nem ismertek. Ezen lépés
hasonlít leginkább a neurális gépi fordítás során használt modellekre. Végül harmadik lépésként az
Assembly-ből kinyert változók és számok behelyettesítésével a második lépésben előállított maszkolt C
kódba visszakapjuk az eredeti C kódot.

A program futtatása során lehetőség van a fentiek kipróbálására grafikus és konzolos felületen is,
valamint saját adatokkal a neurális hálók tanítására.

A program további fejlesztésére még sok lehetőség van, egyrészt a használt nyelvtan bővítése
(pl. \texttt{if-else}, \texttt{while} utasítások, függvényhívások stb.), másrészt a pontosság
növelése érdekében komplexebb modellek használata.

